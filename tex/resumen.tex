% !TeX root = ../main/main.tex
\documentclass[../main/main.tex]{subfiles}

\begin{document}

  \begin{center}
    \textbf{\large Resumen}
  \end{center}

  En esta monografía se analizó el proceso de minería que utiliza ``Graboid'', el primer gusano utilizado para minería de criptomonedas que funciona en el el entorno de virtualización de contenedores de aplicaciones Docker y que al momento de la presentación de este documento se encuentra en un porcentaje considerable de imágenes provistas oficialmente en el servidor oficial de DockerHub. Este gusano utiliza ciclos de procesador y un porcentaje de memoria RAM mediante scripts en lenguaje javascript que son ejecutados en el navegador web del cliente, con lo cual se presentan problemas de rendimiento en el cliente mientras se ejecuta la minería de criptomonedas.

  Para trazar el proceso de infección se utilizó un servidor en un entorno controlado, en el cual se instaló Docker y se utilizó una imagen identificada por su infección.

  En cuanto al cliente, se realizaron pruebas en los dos navegadores web más populares, Google Chrome y Mozilla Firefox, para posteriormente realizar el análisis de procesamiento ocupado por el gusano.

  Para demostrar la validez del análisis se utilizó un sistema operativo ``limpio'' recién instalado con la mínima cantidad de procesos corriendo por detrás. En este entorno controlado se realizaron las mediciones del uso del procesador y la memoria RAM a fin de encontrar el porcentaje de uso del proceso generado por el gusano desde el navegador web.

  En base a los datos obtenidos después de las pruebas se muestran los resultados y posterior a su análisis se llegan a obtener las conclusiones.

  Sin duda, el primer gusano creado con este paradigma, Graboid, plantea una infección que en un futuro podría llegar a popularizarse; de hecho, al momento de la presentación de este documento podrían existir otras infecciones de este tipo, de las cuales, solo el tiempo se encargará de informarnos de su existencia.

\end{document}