\documentclass[../main.tex]{subfiles}

\begin{document}
\espacio

  En este capítulo se muestran: el panorama general del problema que se desea solucionar, la importancia teórica y práctica del desarrollo de la solución, el método empleado en el trabajo y el alcance que tiene la solución desarrollada.

  \section{Problemática}

  El paradigma computacional ha cambiado bastante desde la comercialización de la computadora personal (PC). De acuerdo a la ley de Moore: ``el número de componentes de un circuito integrado seguirá doblándose cada año, y en 1975 serán mil veces más complejos que en 1965'' \cite{articulo:ley_de_moore}.
  
  Sin embargo actualmente, con la popularidad que alcanzaron los conceptos de Dispositivos Inteligentes (Smart Devices), Dispositivos Portátiles (Wearables) y el Internet de las Cosas (IOT), se observa claramente que los circuitos integrados han llegado a un límite de tamaño tan pequeño que es muy dificil de reducir desde hace algunos años.
  
  Para solventar tal limitación es que los fabricantes de microprocesadores cambiaron el paradigma de desarrollo de hardware de la arquitectura mono-núcleo a la arquitectura multi-núcleo. Posterior a este cambio se pudo observar que el límite del número de procesadores de uso general ha llegado también a un límite difícil de superar debido al calentamiento al que son sometidos los circuitos integrados dentro de dichos procesadores ya que la temperatura es inversamente proporcional al tamaño de los dispositivos y al trabajo de cómputo que se designa a cada circuito; por ello muchas aplicaciones actuales hacen uso de recursos definidos para tareas específicas, por ejemplo para el cómputo de bloques muy grandes de imágenes o videos, se utilizan tarjetas gráficas que procesan volúmenes gigantescos de datos para entregar el resultado nuevamente al procesador central o bien para mostrar el resultado por pantalla u otro dispositivo de salida.

  Debido a la carga de tareas que se asigna a la CPU, ésta puede llegar a formar colas insostenibles de procesos, por lo tanto, se intentará demostrar la factibilidad de la distribución de tareas a los procesadores de la Unidad de Procesamiento Gráfico (GPU) para incrementar la velocidad de ejecución del algoritmo Estándar de Encriptación Avanzado (AES).
  
  \section{Importancia teórica y práctica}
  



  \bibliografia
\end{document}