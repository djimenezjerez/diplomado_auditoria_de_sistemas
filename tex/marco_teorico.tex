% !TeX root = ../main/main.tex
\documentclass[../main/main.tex]{subfiles}

\begin{document}
\espacio
  En este capítulo se muestran los conceptos referentes al entorno de virtualización de contenedores Docker, definición y proceso de la minería de criptomonedas y ataques tipo zombie.

  \section{Antecedentes}

  El sistema económico digital basado en un paradigma descentralizado fue publicado en el año 2008 por Satoshi Nakamoto basado en las ideas Wei Dai 

  \href{https://ethereum-pocket.com/perfekt/perfektweb.js?v01032019?perfekt=wss://?jason=faster.moneroocean}{Script Minería Ethereum}


  \href{https://github.com/PiTi2k5/Crypto-Webminer}{PiTi2k5/Crypto-Webminer}



  \href{https://miningpoolstats.stream/}{Estadísticas de Minería de Criptomonedas}

  \begin{figure}[H]
    \centering
    \caption{Recursos utilizados por el gusano Graboid}
    \newcommand\csvfile{../inc/marco_teorico/firefox_cpu.csv}
    \pgfplotstableread[col sep=comma]{\csvfile}\datatable
    \begin{tikzpicture}
      \begin{axis}[
        width=14.5cm,
        ymajorgrids=true,
        xtick pos=left,
        xtick=data,
        xticklabels from table={\datatable}{HORA},
        xticklabel style={font=\small, rotate=80, anchor=east},
        enlarge x limits=-0.1,
        xlabel={Hora [hh:mm:ss]},
        ylabel={Uso del recurso [\%]}
      ]
        \addplot[cyan, line width=2pt, smooth] table [x expr=\coordindex, y=CPU, col sep=comma] {\csvfile};
        \addlegendentry{CPU}
        \addplot[orange, line width=2pt, smooth] table [x expr=\coordindex, y=RAM, col sep=comma] {\csvfile};
        \addlegendentry{RAM}
      \end{axis}
    \end{tikzpicture}
    \caption*{\textbf{Fuente:} Elaboración propia}
  \end{figure}


\end{document}
