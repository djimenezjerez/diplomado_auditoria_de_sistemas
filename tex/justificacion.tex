\documentclass[../main.tex]{subfiles}

\begin{document}
\espacio

  En este capítulo se muestran la justificación del problema con una visión desde diferentes ámbitos o enfoques de acuerdo a los paradigmas tecnológico y científico.

  \section{Justificación tecnológica}

  Como se mencionó anteriormente, de acuerdo al paradigma actual del desarrollo de hardware de microprocesadores, se llegó a un límite difícil de superar con los materiales de fabricación actuales.

  Intel propuso el modelo de desarrollo de hardware ``Tick-Tock'' que consiste en un lanzamiento cada 18 meses (1 año y medio), ambas palabras hacen referencia a:

  \begin{description}
    \item[Tick:] Mejoría de una arquitectura anterior
    \item[Tock:] Lanzamiento de una nueva arquitectura
  \end{description}

  Pero si se realiza una comparación del último ``Tock'' que realizó Intel con respecto al microprocesador de la gama i7, se obtiene un resultado claro, y es que en 6 años se redujo el tamaño de transistor de 22nm a 14nm, duplicando los procesadores para llegar a un total de 8 procesadores físicos del CPU Intel i7-3770 del año 2012 al procesador Intel i7-9900 del año 2018. Pero la frecuencia se incrementó tan solo en 2MHz, por lo tanto se puede concluir que el paradigma tiene como objetivo llegar a multiplicar el número de procesadores pero no así la frecuencia de trabajo de los mismos.

  \begin{table}[]
    \begin{tabular}{|p{5.6cm}|p{4cm}|p{4cm}|}
      \hline
      \multicolumn{1}{|c|}{\textbf{Característica}} & \textbf{I7-9900K} & \textbf{I3-3770} \\ \hline
      Núcleos & 8 & 4 \\ \hline
      Hilos & 16 & 8 \\ \hline
      Serie & Coffee Lake & Ivy Bridge \\ \hline
      Socket & FCLGA1151 & FCLGA1155 \\ \hline
      Fecha de lanzamiento & 4º trimestre de 2018 & 2º trimestre de 2012 \\ \hline
      Cache & 16 MB & 8 MB \\ \hline
      Set de instrucciones & Intel® SSE4.1, Intel® SSE4.2, Intel® AVX2 & SSE4.1/4.2, AVX \\ \hline
      Litografía & 14 nm & 22 nm \\ \hline
      Velocidad de bus & 8 GT/s DMI3 & 5 GT/s DMI \\ \hline
      Solución térmica & PCG 2015D (130W) & 2011D \\ \hline
      Máximo tamaño de memoria & 64 GB & 32 GB \\ \hline
      Tipo de memoria & DDR4-2666 & DDR3 1333/1600 \\ \hline
      Ancho de banda de memoria & 41.6 GB/s & 25.6 GB/s \\ \hline
      Frecuencia base para gráficos & 350 MHz & 650 MHz \\ \hline
      Frecuencia máxima para gráficos & 1.20 GHz & 1.15 GHz \\ \hline
      Configuración de PCI Express & 1x16, 2x8, 1x8+2x4 & 1x16, 2x8, 1x8 \& 2x4 \\ \hline
    \end{tabular}
  \end{table}

  Por lo tanto se justifica el uso de la Unidad de Procesamiento Gráfico para la distribución de tareas repetitivas aptas para un enfoque de ejecución en paralelo, ya que de acuerdo a la tabla comparativa anterior se observa que el paradigma de desarrollo de hardware por parte de los fabricantes es hacia un ecosistema de procesadores en paralelo en lugar de un bajo número de procesadores trabajando a frecuencias altas.

  \section{Justificación científica}

  En lo referente al ámbito científico, la utilidad de esta investigación radica en la profundización del estudio acerca de los métodos de ejecución de aplicaciones en múltiples hilos utilizando la Unidad de Procesamiento Gráfico (GPU), con la finalidad de llegar a utilizar las mallas de procesadores disponibles en las tarjetas gráficas, este estudio a la fecha de realización de la presente investigación, aún no se ha profundizado, ni es de aplicación general.

  \bibliografia
\end{document}